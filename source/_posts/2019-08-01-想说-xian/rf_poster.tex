% fancytikzposter.tex, version 2.1
% Original template created by Elena Botoeva [botoeva@inf.unibz.it], June 2012
% 
% This file is distributed under the Creative Commons Attribution-NonCommercial 2.0
% Generic (CC BY-NC 2.0) license
% http://creativecommons.org/licenses/by-nc/2.0/ 


\documentclass[a0,portrait]{a0poster}
\usepackage{graphicx}%插入图片支持
\usepackage{float}%浮动图片
\DeclareGraphicsExtensions{.eps,.ps,.jpg,.bmp,.png,.svg}%支持图片格式
\usepackage{fancytikzposter} 

%%%%% --------- Change here if you want ---------- %%%%%
%% margin for the geometry package, must be changed before using the geometry package
%% default value is 4cm
 \setmargin{2.5}

%% the space between the blocks
%% default value is 2cm
\setblockspacing{0.6}

%% the height of the title stripe in block nodes, decrease it to save space
%% default value is 3cm
% \setblocktitleheight{3}

%% the number of columns in the poster, possible values 2,3
%% default value is 2
% \setcolumnnumber{3}

%% the space between two or more groups of authors from different institutions
%% used in \maketitle
% \setinstituteshift{10}

%% which template to use
%% N1 simple, standard look, with a colored background and gray boxes
%% N2 board with nodes
%% N3 another standard look
%% N4 envelope-like look
%% N5 with a wave-like head, original idea taken from
%%%% http://fc09.deviantart.net/fs71/f/2010/322/1/1/scientific_poster_by_nabuy-d333ria.jpg
\usetemplate{2}

%% components of the templates
%% (the maximal possible numbers are mentioned as the parameters)
% \usecolortemplate{4}
% \usebackgroundtemplate{5}
% \usetitletemplate{5}
% \useblocknodetemplate{5}
% \useplainblocktemplate{4}
% \useinnerblocktemplate{2}


%% the height of the head drawing on top 
%% applicable to templates N3, 4 and 5
% \setheaddrawingheight{14}


%% change the basic colors
%\definecolor{myblue}{HTML}{008888} 
%\setfirstcolor{myblue}% default 116699
%\setsecondcolor{gray!80!}% default CCCCCC
%\setthirdcolor{red!80!black}% default 991111

%% change the more specific colors
% \setbackgrounddarkcolor{colorone!70!black}
% \setbackgroundlightcolor{colorone!70!}
% \settitletextcolor{textcolor}
% \settitlefillcolor{white}
% \settitledrawcolor{colortwo}
% \setblocktextcolor{textcolor}
% \setblockfillcolor{white}
% \setblocktitletextcolor{colorone}
% \setblocktitlefillcolor{colortwo} %the color of the border
% \setplainblocktextcolor{textcolor}
% \setplainblockfillcolor{colorthree!40!}
% \setplainblocktitletextcolor{textcolor}
% \setplainblocktitlefillcolor{colorthree!60!}
% \setinnerblocktextcolor{textcolor}
% \setinnerblockfillcolor{white}
% \setinnerblocktitletextcolor{white}
% \setinnerblocktitlefillcolor{colorthree}




%%% size of the document and the margins
%% A0
%\usepackage[margin=\margin cm, paperwidth=118.9cm, paperheight=84.1cm]{geometry} 
%\usepackage[margin=\margin cm, paperwidth=84.1cm, paperheight=118.9cm]{geometry}
%% B1
% \usepackage[margin=\margin cm, paperwidth=70cm, paperheight=100cm]{geometry}
%\usepackage[margin=\margin cm, paperwidth=118.9cm, paperheight=100.1cm]{geometry} 


%% changing the fonts
\usepackage{cmbright}
%\usepackage[default]{cantarell}
%\usepackage{avant}
%\usepackage[math]{iwona}
\usepackage[math]{kurier}
\usepackage[T1]{fontenc}


%% add your packages here
%\usepackage{hyperref}  % % DISABILITATO I VIRTUAL REF


% % % DAN ADDED
\newtheorem{thm3}{Proposition}

% % % END DAN ADDED

\title{RF spectrum of strongly interacting bose gase across d-wave resonance} %Networking-Computing resource allocation for Hard Real-Time Green Cloud applications}
%\author{Elena Botoeva\\
%  KRDB Research Centre, Free University of Bozen-Bolzano, Italy\\
%  \texttt{botoeva@inf.unibz.it}
%}
\author{Ran Qi; Zeqing Wang\\ %\vspace{20pt}
	Department of Physics, Renmin University of China \\
	}

\begin{document}

%%%%% ---------- the background picture ---------- %%%%%
%% to change it modify the macro \BackgroundPicture
\ClearShipoutPicture
\AddToShipoutPicture{\BackgroundPicture}
\noindent % to have the picture right in the center
\begin{tikzpicture}
  \initializesizeandshifts
  % \setxshift{15}
  % \setyshift{2}


  %% the title block, #1 - shift, the default value is (0,0), #2 - width, #3 - scale
  %% the alias of the title block is `title', so we can refer to its boundaries later
  \ifthenelse{\equal{\template}{1}}{ 
    \titleblock{47}{1}
  }{
    \titleblock{47}{1.5}
  }

  %% a logo can be added to the title block
  %% #1 - anchor relative to the title block, #2 - shift, #3 - width, #3 - file name
   \ifthenelse{\equal{\template}{2}}{ 
     %\addlogo[south west]{(2,0)}{6cm}{unibz_b.png}
     \addlogo[south west]{(0,0)}{7cm}{logos/ieee-comsoc.png} 
        \addlogo[south west]{(68,0)}{3cm}{logos/hhh.png} }{
     \addlogo[south west]{(0,0.2)}{11cm}{logos/ieee-comsoc.png} 
   }

  %% a block node, with the specified position (optional), title and the content
  %% #1 - where (optional), #2 - title, #3 - text
  %%%%%%%%%% ------------------------------------------ %%%%%%%%%%

%  	\normalsize 
%       \Large
%       \large
% 	\normalsize
  
  
  \vspace{80pt}
%  \blocknode%
  \blocknodew[($(currenty)-(0, 2)$)]{39}{Introduction} %
%  {Introduction}%
  { 
    \Large
      	Compared to the general solid state system, the interaction
        can be easily tuned by Feshbach resonance in cold atom
        systems. Scattering resonances now enable resonance of
        different partial waves. For the fermion superfluid, the most
        studied in the past is the s-wave resonance.
        \\
        In recent years, degenerate quantum gases near
        high-partial have been realized. Degenerate quantum gases
        near the high-partial wave, of the p-wave and d-wave have been
        realized. For p-waves, Christopher Luciuk et al.'s experiments
        were published in Nature Physics [3] in 2016, and Zhenhua Yu
        [4] et al. also proposed relevant theories. The related work
        of d-wave has been achieved [5][6]. \\
        Nowadays, people are paying more and more attention to the
        quantum gas near the high-partial wave, because the
        high-partial-wave interaction has many different characteristics
        compared to the s-wave interaction. For example,
        high-partial-wave interactions are anisotropic. For d-waves,
        some topological states can be achieved. \\
        RF spectrum is very helpful in understanding the
        interactions in cold atom systems. For example, in 2004, the
        RF spectrum experimentally strengthened the establishment of
        the BEC-BCS crossover for the fermion multi-body pairing in
        the strong interaction Fermi gas. By measuring the RF spectrum
        of the system at different temperatures, it is found that
        there is a shift in the RF spectrum at low temperatures, thus
        establishing the formation of pairing in the system.
        \quad \\
        \quad \\
  }


  \blocknode{RF spectrum measurement process}
  {
    \Large
    It is assumed that an atom has three different internal
    states, which may be hyperfine structures of the
    same atom. Two of the states have interactions, and the
    other has no interaction with the two states.  \\
           There are generally two approaches in experiments. One
           is the initial state preparation in two interacting states,
           the initial state may is a superfluid state, and then one
           of the states is hit to another state that is not
           interacting, experimentally measured after a period of
           time. It can be shown by linear response theory that the
           number of particles on the state that has no 
           interaction is
           related to the Green's function. There is also a case where
           the initial state is prepared in two states that do not
           interact, and then hit go to another state. \\

           \quad \includegraphics[width=15cm]{./images/RF.eps}
           \quad\quad\quad
           \includegraphics[width=15cm]{./images/exp.eps}
           \\
           \quad \\
  }
  
   %%%%%%%%%% ------------------------------------------ %%%%%%%%%%
   %\blocknodew[($(currenty)-(3.5,0)$)]{30}{Variable Width Block Nodes} %
 	
	\startsecondcolumn
        %%%%%%%%%% ------------------------------------------ %%%%%%%%%%
         \blocknodew[($(currenty)-(0, 2)$)]{39}{} %
	 {
           \Large


    

           For example, Regal C A et al. [1] worked for $^{40}K$ in
           2003. The initial state is prepared in the two hyperfine
           states of $^{40}K$ $|F=9/2, m_F =-9/2\rangle$ and $|9/2,
           -7/2\rangle$ , the initial state is not interacting, join
           after the RF pulse, the atom is moved from the state
           $|9/2,-7/2\rangle$ to $|9/2,-5/2\rangle$ , due to Feshbach
           resonance, 
           $|9/2,-5/2\rangle$ and $|9 /2, -9/2\rangle$ has a strong
           interaction.
           \quad \\
                      \quad \\
	}

  %%%%%%%%%% ------------------------------------------ %%%%%%%%%%
  %\blocknodew[($(currenty)-(3.5,0)$)]{30}{Variable Width Block
  %Nodes} %

  \blocknode %
  {Theoretical Method} %
  {\Large
    Theoretically, the calculation of the RF spectrum is based on the 
    linear response theory. It is concluded that the derivative of the
    particle number versus time is related to the Green's function. As
    long as the Green's function is calculated, the corresponding RF
    spectrum can be obtained [1][2].


    
\begin{align}
\frac{\mathrm{d}N_3}{\mathrm{d}t}\propto \mathrm{Im} \int G_3(k,
  -\tau) G_2(k, \tau) e^{-\mathrm{i}(\omega ' + \mathrm{i}0^+)\tau}
  \mathrm{d} \tau
\end{align}

\quad \\
\quad \\


           \quad \includegraphics[width=15cm]{./images/dn.eps}
           \quad \includegraphics[width=15cm]{./images/feynnmanDiagrams.eps}

           \quad \\
           \quad \\

    For the RF spectrum of the d-wave, no one has yet calculated
    it. We intend to use the Ladder
    diagram to calculate the self energy, then calculate the Green's
    function, and then obtain the corresponding RF spectrum, and then
    analyze the RF spectrum characteristics under different magnetic
    field regions near the d-wave resonance. This is the guiding
    significance of the experiment.

    \quad \\
    \quad \\
    
	}


    \plainblock{($(currenty)$)} %-(xshift) + (xshift) -(yshift) %[($(currenty)+(0,10)$)]%
  {38}{Reference} %
  {
    \Large
    \begin{enumerate}
    \item Paivi Torma 2016 Phys. Scr. 91 043006 \\
    \item Junjun Xu, Qiang Gu, and Erich J. Mueller Phys. Rev. A 88,
    023604 (2013) \\
    \item Christopher Luciuk, Stefan Trotzky, Scott Smale, Zhenhua Yu,
    Shizhong Zhang, Joseph H. Thywissen Nature Physics 12, 599–605
    (2016) \\
    \item Zhenhua Yu, Joseph H. Thywissen, and Shizhong Zhang
    Phys. Rev. Lett. 115, 135304\\
    \item Yue Cui, Chuyang Shen, Min Deng, Shen Dong, Cheng Chen, Rong
    Lü, Bo Gao, Meng Khoon Tey, and Li You Phys. Rev. Lett. 119,
    203402 (2017)\\
    \item Xing-Can Yao, Ran Qi, Xiang-Pei Liu, Xiao-Qiong Wan Yu-Xuan
    Wang, Yu-Ping Wu, Hao-Ze Chen, Peng Zhang, Hui Zhai, Yu-Ao Chen,
    and Jian-Wei Pan arXiv:1711.06622
    \end{enumerate}
  }
  
\end{tikzpicture}


\end{document}




