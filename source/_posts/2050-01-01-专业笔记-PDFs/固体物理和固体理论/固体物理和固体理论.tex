\documentclass{ctexart}
\usepackage{amsmath}

\title{固体物理和固体理论}
\author{ZQW}


\begin{document}

\maketitle


\section{Sommerfeld Theory}








\section{Bravais 格子}

Bravais 格子的判别方法:

站在任意格点上, 通过周围格子的方位与距离关系, 不能知道在不同的格点上!

例如, honeycomb 格子就可以通过周围格子的方位与距离关系判断出两种
不同的格点, 因此 honeycomb 格子不是 Bravais 格子.

\section{倒格子}

\subsection{Definition of Reciprocal Lattice}

Aschroft 书 86 页给出了倒格子的定义:

The set of all wave vectors $\vec{K}$ that yield plaves with the
periodicity of a given Bravais lattice is known as the reciprocal
lattice.

倒格子基矢就是为周期性而生, 为 Bravais 格子而生! 倒格子能够产生周期性的平面波.

Aschroft 书86页:

such a set of $\vec{K}$ is called a reciprocal lattice only if
the set of vectors $\vec{R}$ is a Bravais lattice.

为 Bravais 而生!

\subsection{More Details about Reciprocal Lattice}

一般的平面波:
\begin{align*}
  e^{\mathrm{i}\vec{k}\cdot \vec{r}}
\end{align*}
没有周期性. 如果让它有 Bravais 格子的周期性, 应该有什么要求呢? 假设 $\vec{K}$
满足我们的要求, 那 $\vec{K}$ 应该满足:
\begin{align*}
  e^{\mathrm{i} \vec{K}\cdot(\vec{r}+\vec{R})} = e^{\mathrm{i}\vec{K}\cdot \vec{r}}
\end{align*}
也就是
\begin{align*}
  \vec{K} \cdot \vec{R} = 2\pi \cdot \mathrm{integer}
\end{align*}
写成分量的形式:
\begin{align*}
  \vec{K} = k_1 \vec{b}_1 + k_2 \vec{b}_2 +k_3 \vec{b}_3 \\
  \vec{R} = n_1 \vec{a}_1 + n_2 \vec{a}_2 +n_3 \vec{a}_3
\end{align*}
它们的基底正交, 归一到 $2\pi$ , 即
\begin{align*}
  \vec{b}_i \cdot \vec{a}_j = 2\pi \delta_{ij}
\end{align*}

$\vec{R}$ 是正格子的k格点, 所以 $n_i$ 都得是整数. 若要任意的 $\vec{K}, \vec{R}$
 都满足
\begin{align*}
  \vec{K} \cdot \vec{R} = 2\pi \cdot \mathrm{integer}
\end{align*}
那么 $k_i$ 也必须都是整数, 就刚好取到倒格子的格点.

倒格子格点产生的平面波具有正格子的周期性!







\section{Bloch定理}

\subsection{Definition}

单电子在 Bravais 格子中的 Hamiltonian :
\begin{align*}
  H = - \frac{\hbar^2}{2m}\nabla ^2 + U(\vec{r}), \quad U(\vec{r}+\vec{R}) = U (\vec{r})
\end{align*}
的本征态可以选为
\begin{align*}
  \psi_{n\vec{k}} (\vec{r}) = e^{\mathrm{i}\vec{k} \cdot \vec{r}} u_{n\vec{k}}(\vec{r}),
  \quad u_{n\vec{k}} (\vec{r}+\vec{R}) = u_{n\vec{k}}(\vec{r})
\end{align*}
是一个相位乘上一个周期函数的形式.

因为 Hamiltonian 具有平移对称性, 所以平移操作与 Hamiltonian 对易, 具有共同本征态.
$e^{\mathrm{i}\vec{k}\cdot \vec{r}}$ 就是平移操作作用在波函数上所得出的本征值!

\subsection{The Born-von Karman Boundary Condition}

假设总的格点数为 $N = N_1N_2N_3$ , 并且满足周期性边界条件, 那么
\begin{align*}
  \psi(\vec{r}+ N_i \vec{a}_i) = \psi(\vec{r}), \quad i = 1, 2, 3
\end{align*}
则
\begin{align*}
  e^{\mathrm{i} N_i \vec{k}\cdot \vec{a}_i} = 1
\end{align*}
所以
\begin{align*}
  \vec{k} = \sum_{i=1}^3 \frac{m_i}{N_i}\vec{b}_i =\sum_{i=1}^3 m_i\frac{\vec{b}_i}{N_i} , \quad m_i\, \mathrm{integral}
\end{align*}
$\vec{k}$ 的取值把倒格子基矢又进行了细分, 正好细分成正格子的格点个数. 所以有 Ashcroft 书 136 页的话:

the number of allowed wave vectors in a primitive cell of the reciprocal lattice is equal to
the number of sites in the crystal.

在一个倒格子中允许的波矢的个数就是正格子的格点的个数.

\subsection{A Proof of Bloch's theorem}

假设系统满足 Born-von Karman 边界条件.

由于体系有周期性, 可以考虑在 $\vec{k}$ 空间求解 Schrodinger 方程.

任何波函数都可以按平面波展开. 满足周期性边界条件的波函数,
可以用有限个分立的平面波展开(由部分离散的 Fourier Transform 可知). 即
\begin{align*}
  \psi(\vec{r}) = \sum_{\vec{k}} c_{\vec{k}} e^{\mathrm{i}\vec{k}\cdot \vec{r}}
\end{align*}
离散的 $\vec{k}$ 能取到的值如定义中那样. 这里的 $\vec{k}$ 的离散来自于
周期性的边界条件, 与所有的格点的数目 $N$ 有关.

周期势 $U(\vec{r})$ 也可以按平面波展开. 由于具有周期性, 也是用有限个分立
的平面波展开. 即
\begin{align*}
  U(\vec{r}) = \sum_{\vec{K}} U_{\vec{K}} e^{\mathrm{i} \vec{K}\cdot \vec{r}}
\end{align*}
离散能取的值就是倒格子的格点. 这里的 $\vec{K}$ 的离散来自于 Bravais 格子
的周期性, 与 Bravais 格子有关.

注意这里出现的两个不同的周期性. 边界条件导致的周期性和 Bravais 格子的周期性.
它们分别导致了 $\vec{k}$ 和 $\vec{K}$ 取值的离散.

然后将展开的结果代入 Schrodinger 方程中有
\begin{align*}
  \left[ -\frac{\hbar^2}{2m}\nabla^2 + \sum_{\vec{K}} U_{\vec{K}} e^{\mathrm{i} \vec{K}\cdot \vec{r}} \right]
  \sum_{\vec{k}} c_{\vec{k}} e^{\mathrm{i}\vec{k}\cdot \vec{r}} = \mathcal{E} \sum_{\vec{k}} c_{\vec{k}} e^{\mathrm{i}\vec{k}\cdot \vec{r}}\\
  \Downarrow \\
  \sum_{\vec{k}}\left\{ \left( -\frac{\hbar^2}{2m}k^2 -\mathcal{E} \right)c_{\vec{k}} +\sum_{\vec{K}}U_{\vec{K}}c_{\vec{k}-\vec{K}} \right\}
  e^{\mathrm{i}\vec{k}\cdot \vec{r}} = 0
\end{align*}
因为 $\vec{k}$ 对于 $\vec{r}$ 积分是正交的, 所以两边左乘 $e^{\mathrm{i}\vec{k}'\cdot \vec{r}}$
然后对 $\vec{r}$ 积分可得
\begin{align*}
  \sum_{\vec{k}} \left\{ \left( -\frac{\hbar^2}{2m}k^2 -\mathcal{E} \right) c_{\vec{k}} +
  \sum_K U_{\vec{K}} c_{\vec{k}-\vec{K}} \right\} \delta_{\vec{k},\vec{k}'} = 0 \\
  \Downarrow \\
   \left( -\frac{\hbar^2}{2m}k^2 -\mathcal{E} \right) c_{\vec{k}} +
  \sum_K U_{\vec{K}} c_{\vec{k}-\vec{K}}  = 0
\end{align*}
通过周期性, 对 $\vec{k}$ 空间的 Schrodinger 方程进行了化简. 发现波函数 $\psi$ 在 $\vec{k}$
空间的展开系数并不是全部耦合在一起的, 只有相差 $\vec{K}$ 才会相互耦合.

现在可以将一开始的展开化简
\begin{align*}
  \psi(\vec{r}) = \sum_{\vec{k}} c_{\vec{k}} e^{\mathrm{i}\vec{k}\cdot \vec{r}}
\end{align*}
为
\begin{align*}
  \psi(\vec{r}) = \sum_{\vec{k}\in F.B.Z.} \psi_{\vec{k}}(\vec{r}), \quad
  \psi_{\vec{k}}(\vec{r}) = \sum_{\vec{K}} c_{\vec{k}-\vec{K}}e^{\mathrm{i}(\vec{k}-\vec{K})\cdot \vec{r}}
\end{align*}
而 $\psi_{\vec{k}}(\vec{r})$ 是本征态. 变换一下形式就是 Bloch's Theorem 定义
中的形式
\begin{align*}
  \psi_{\vec{k}}(\vec{r}) = e^{\mathrm{i}\vec{k}\cdot \vec{r}} \sum_{\vec{K}}c_{\vec{k}-\vec{K}}e^{-\mathrm{i}\vec{K}\cdot \vec{r}}
\end{align*}

\subsection{讨论}

因为晶体不具有完全的平移对称性, 只是具有具有沿格矢的平移对称性, 所以
动量 $\vec{p}$ 不是一个好的量子数. 这里的 $\vec{k}$ 不是常说的那个"动量".

$\vec{p}$ 和 $\vec{k}$ 是相似的. 前者表征完全的平移对称性, 而后者表征晶体
中的平移对称性.

\subsection{Band Index}

将 Bloch 波函数代回到 Schrodinger 方程中进行化简得到
\begin{align*}
  \left( \frac{\hbar^2}{2m}\left( \frac{1}{\mathrm{i}}\nabla +\vec{k} \right)^2
  + U(\vec{r})\right)u_{\vec{k}}(r) = \mathcal{E}_{\vec{k}}(\vec{r}) u_{\vec{k}}(\vec{r})\\
  \mathrm{Boundary Condition:} \quad u_{\vec{k}}(\vec{r}) = u_{\vec{k}}(\vec{r}+\vec{R})
\end{align*}
一个在周期性边界条件下的本征值问题, 它应该有分立的本征值, 用指标 $n$ 来标记,
这就是 Band Index !

当 $\vec{k}$ 的间隔非常小的时候, 是趋于连续的. 由上述方程也可以看出 $\vec{k}$
作为方程的参数, 其解的 $n$ 值固定的时候, 是 $\vec{k}$ 的函数.


\section{参考文献}

Neil W. Ashcroft, N David Mermin, Solid State Physics

黄昆, 韩汝琦, 固体物理学















\end{document}

%%% Local Variables:
%%% mode: latex
%%% TeX-master: t
%%% End:
