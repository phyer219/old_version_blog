\documentclass{ctexart}
\usepackage{amsmath}
\title{复变函数}
\author{ZQW}


\begin{document}
\maketitle

\section{解析函数}
\section{解析( Anlyticity )}

\subsection{惯用记法}

任意复数 $z$ 记为
\begin{align*}
  z = x + \mathrm{i}y
\end{align*}

复函数 $f(z)$ 记为
\begin{align*}
  f(z) = u(x,y) +\mathrm{i} v(x,y)
\end{align*}

例如
\begin{align*}
  f(z) = z + z^2 = x + \mathrm{i}y + (x + \mathrm{i}y)^2 = x + x^2 - y^2 + \mathrm{i}(y + 2xy)
\end{align*}
其中
\begin{align*}
  u(x,y) =& x + x^2 - y^2 \\
  v(x,y) =& y + 2xy
\end{align*}

\subsection{导数}

由于
\begin{align*}
  f(z) = f(x,\mathrm{i}y)
\end{align*}
可以看成是一个二元函数.
那么有类似于一般的二元函数梯度
\begin{align*}
  \nabla g(x,y) = \frac{\partial}{\partial x} g(x,y) \vec{i}+ \frac{\partial}{\partial y} g(x,y) \vec{j}
\end{align*}
的定义, 复函数的导数包含两个正交的方向上的导数
\begin{align*}
  \frac{\partial}{\partial x} f(x,\mathrm{i}y) =&\frac{\partial u}{\partial x} + \mathrm{i}\frac{\partial v}{\partial x}\\
  \frac{\partial}{\partial y} f(x,\mathrm{i}y) =&\frac{\partial u}{\partial \mathrm{i}y} + \mathrm{i} \frac{\partial v}{\partial \mathrm{i}y}
                                               =- \mathrm{i}\frac{\partial u}{\partial y} +\frac{\partial v}{\partial y}
\end{align*}
解析 ( Anlyticity ) 函数的定义是满足以下柯西-黎曼条件 (Cauchy-Riemann conditions)
\begin{align*}
  \frac{\partial u}{\partial x} =&\quad \frac{\partial v}{\partial y} \\
  \frac{\partial u}{\partial y} =& -\frac{\partial v}{\partial x}
\end{align*}
的复函数. 可以看出, 对于解析函数, 在 $x$ 方向和在 $\mathrm{i}y$ 方向上的导数是一样的.

\subsection{解析函数一定可以写在 $z$ 的函数}

一个复函数, 即可以看成是 $(x,y)$ 的函数, 也可以看成是 $(z,z^{*})$ 的函数.
因为这是两组独立的变量, 变换关系是
\begin{align*}
  \left\{
  \begin{array}{c}
    z = x + \mathrm{i} y \\
    z^{*} = x - \mathrm{i}y
      \end{array}
  \right.
  \quad  \quad  \quad  \quad
    \left\{
  \begin{array}{c}
    x = \frac{1}{2}(z + z^{* }) \\
    y = \frac{1}{2\mathrm{i}} (z - z^{*})
          \end{array}
  \right.
\end{align*}

所以, 复函数更加普遍的定义应该写成 $f(z,z^{* })$ , 但是对于解析函数有
\begin{align*}
  \frac{\partial}{\partial z^{* }}f(x, y) =& \frac{\partial f}{\partial x}\frac{\partial x}{\partial z^{* }}
               +\frac{\partial f}{\partial y}\frac{\partial y}{\partial z^{* }} \\
        =& \left(\frac{\partial u}{\partial x} + \mathrm{i}\frac{\partial v}{\partial x}\right)\left(\frac{1}{2}\right)
               + \left(\frac{\partial u}{\partial y} +\mathrm{i}\frac{\partial v}{\partial y}\right)\left(-\frac{1}{2\mathrm{i}}\right) \\
          =&0
\end{align*}
所以解析函数不包含 $z^{* }$ .


\section{留数定理}

假设一函数函数可以写成
\begin{align*}
  g(z) = \frac{f(z)}{z - z_0}
\end{align*}
的形式, 其中 $f(z)$ 没有奇点. 那么它在 $z = z_0$ 处有奇点.对任意包含 $z_0$ 点的路径 $C$ 逆时针积分
\begin{align*}
  \oint_{C}\frac{f(z)}{z -z_0} \mathrm{d}z
\end{align*}
与在奇点附近上一个无穷小的圆的积分是相等的
\begin{align*}
  \oint_{C_0}\frac{f(z)}{z -z_0} \mathrm{d}z
\end{align*}
用振幅和辐角来表示 $z-z_{0} = r e^{\mathrm{i}\phi}$ , 因为在圆上积分, 所以 $r$ 是常数, 那么有
\begin{align*}
  \oint_{C_0}\frac{f(z)}{z -z_0} \mathrm{d}z &= \int_0^{2\pi}\frac{f(z)}{r e^{\mathrm{i}\phi}} \mathrm{i}re^{\mathrm{i}\phi}\mathrm{d}\phi \\
          &=\mathrm{i}\int_0^{2\pi} f(z_0) \mathrm{d}\phi \\
           &= \mathrm{i} 2\pi f(z_{0})
\end{align*}
所以最终就有
\begin{align*}
  \oint_{C}g(z)\mathrm{d}z = \oint_{C}\frac{f_z}{z -z_0}\mathrm{d}z = 2\pi \mathrm{i} f(z_0) = 2\pi \mathrm{i}\cdot g(z)(z-z_{0})|_{z=z_0}
\end{align*}
$g(z)(z-z_0)|_{z=z_0}$ 就是留数, 去掉奇点留下的数.

推广到多个奇点的情况
\begin{align*}
  \oint_C g(z) \mathrm{d}z = 2\pi \mathrm{i} \sum_{k=1}^N \mathrm{Res}[g(z_k)]
\end{align*}

\section{参考文献}

Jim, Napolitano, March 9, 2013, Notes on Complex Analysis in Physics

http://www.rpi.edu/dept/phys/Courses/PHYS6520/Spring2018/NotesOnComplexAnalysis.pdf

\end{document}
%%% Local Variables:
%%% mode: latex
%%% TeX-master: t
%%% End:
